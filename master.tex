\documentclass[a4paper,10pt]{amsart}
\usepackage{amsthm}

%opening
\title{Counting points on elliptic curves}
\author{Ole Andre Birkedal}

\begin{document}
\newtheorem{thm}{Theorem}
\newtheorem{mydef}{Definition}
\newtheorem{ex}{Example}
\newtheorem{prop}{Proposition}
\newtheorem{lemma}{Lemma}

\begin{abstract}
hei hei
\end{abstract}

\maketitle
%\tableofcontents


\section{Algebraic geometry}
In this section we define the fundamental objects in algebraic geometry and state
some facts about their structure. We will then move on to the theory of
curves and Weil divisors.

\begin{mydef}
\emph{Projective n-space} over a field $k$ denoted $\mathbb{P}^n$ is the set 
of all $(n+1)$-tuples $$(x_0,\ldots,x_n)\in\mathbb{A}^{n+1}$$
modulo the equivalence relation given by $(x_0,\ldots,x_n)\sim(y_0,\ldots,y_n)$ 
if there exists $\lambda\in k$ such that $x_i=\lambda y_i$.
The equivalence class $\{(x_0,\ldots,x_n)\}$ is denoted $[x_0,\ldots,x_n]$.
Here $\mathbb{A}^n = \{ (x_1,\ldots,x_n) : x_i \in \bar{k} \} $ is the affine $n$-space.
\end{mydef}

Let $Gal(\bar{k}/k)$ be the galois group of $\bar{k}/k$. This group acts on
$\mathbb{A}^n$, such that when $\sigma \in Gal(\bar{k}/k)$ and $P\in \mathbb{A}^n$
we define $\sigma(P) = (\sigma(x_1),\ldots,\sigma(x_n))$. Now we define
the set of $k$-rational points in $\mathbb{A}^n$ to be those fixed under action by
the galois group
$$ \mathbb{A}^n(k) = \{ P \in \mathbb{A}^n : \sigma(P) = P\, \forall\, \sigma \in
Gal(\bar{k}/k) \} $$

Similarly we define the set of $k$-rational points in $\mathbb{P}^n$ to be
$$ \mathbb{P}^n(k) = \{ P \in \mathbb{P}^n : \sigma(P) = P\, \forall\, \sigma \in 
Gal(\bar{k}/k) \} $$

\begin{mydef}
 A polynomial $f\in\bar{k}[X]$ is said to be \emph{homogeneous of degree $d$} if for all
$\lambda\in\bar{k}$ we have.
$$f(\lambda x_0,\ldots,\lambda x_n) = \lambda^d f(x_0,\ldots,x_n)$$
Furthermore an ideal $I\subseteq\bar{k}[X]$ is said to be homogeneous if it is generated
by homogeneous polynomials.
\end{mydef}

\begin{mydef}
 A \emph{projective algebraic set} is of the form
$$ V_I = \{ P\in \mathbb{P}^n : f(P) = 0\, \forall homogeneous\, f\in I \} $$
Given such a set $V$ we associate to it an ideal $I(V) \in \bar{k}[X]$ generated by
$$ \{f\in\bar{k} : f\, homogeneous\, and\, f(P)=0\, \forall P\in V \} $$
\end{mydef}

\begin{mydef}
 A projective algebraic set is called a \emph{projective variety} if the homogeneous
ideal defined above is a prime ideal in $\bar{k}[x]$.
\end{mydef}

\begin{mydef}
 Let $V/k$ be a projective variety (i.e. V defined over $k$), then the projective coordinate
ring of $V/k$ is defined by
$$ k[V] = \frac{k[x]}{I(V/k)}$$
Note that since $I(V/k)$ is a prime ideal, the coordinate ring is an integral domain.
This enables us to form the quotient field of $k[V]$ which we denote $k(V)$, and it is called
the \emph{function field} of $V$.
\end{mydef}

A rather interesting ideal to keep in mind is given by
$$ M_p = \{ f\in \bar{k}[V] : f(P)=0 \} $$
This is a maximal ideal because the map $\phi: \bar{k}[V] \rightarrow \bar{k}$ given by
$ f \mapsto f(P) $ has kernel exactly $M_p$. It is clearly onto, so it induces an
isomorphism $$\tilde{\phi}: \bar{k}[V]/M_p \rightarrow \bar{k} $$

\begin{mydef}
 The \emph{localization of $\bar{k}[V]$ at $M_p$} is given by
$$ \bar{k}[V]_P = \{ h \in \bar{k}[V] : h = f/g\, f,g\in \bar{k}[V]\, and\, g(P)=0 \} $$
The functions in $\bar{k}[V]_P$ are all defined at $P$.
\end{mydef}

\begin{ex}
 If $V$ is a variety given by a single non-constant polynomial equation
$$f(x_1,\ldots,x_n) = 0$$ 
then the dimension of the variety $dim(V)$ is $n-1$. The (projective) varieties
we will study are called \emph{elliptic curvevs} and are
given by polynomial equations
$$E: y^2 = x^3+ax+b$$
They correspond to polynomials of the form $f(x,y) = x^3+ax+b-y^2$ so $dim(E)=1$.
We say curves are projective varieties of dimension $1$.
\end{ex}

The objects we will be working on are projective varieties, but they are not
very interesting unless we define maps between them.

\begin{mydef}
 Let $V_1$ and $V_2$ be projective varieties, a \emph{rational map} $\phi: V_1 \rightarrow V_2$
is a set of maps $\{\phi_0,\ldots,\phi_n\}$ with $\phi_i \in \bar{k}(V_1)$ such that for every
$P\in V_1$ we define
$$\phi(P) = [\phi_0(P),\ldots,\phi_n(P)] \in V_2$$
Such a rational map is called a \emph{morphism} if it is defined at every point $P$.
\end{mydef}

The varieties and the morphisms between them make up a category, so our next
definition of an isomorphism will be the general one found in category theory.

\begin{mydef}
 Two varieties $V$ and $W$ are \emph{isomorphic} denoted $V\simeq W$
if there exist morphisms $\phi: V \rightarrow W$ and $\psi: W \rightarrow V$ such that
$\phi \psi = 1_W$ and $\psi \phi = 1_V$.
If the rational functions $\psi$ and $\phi$ are defined over $k$ we say that $V$ and $W$
are isomorphic over $k$. If not, they are isomorphic over some field extension of $k$
(i.e. $\bar{k}$).
\end{mydef}

\subsection{Curves and divisors}
Recall that curves are projective varieties of dimension one. Even more special
are elliptic curves, which are curves with \emph{genus} equal to 1. This will
be introduced later on. These are in practise the only curves we will be working with.

\begin{mydef}
 Let $C$ be a curve and $P\in C$ a non-singular point on the curve. A valuation on
$\bar{k}[C]_P$ is given by
$$ ord_P : \bar{k}[C]_P \rightarrow \{ 0, 1, 2, \ldots \} \cup \{ \infty \} $$
$$ ord_P(f) = max \{ d\in \mathbb{Z} : f\in M_P^d \} $$
This is called \emph{the order of $f$ at $P$}.
Letting $ord_P(f/g) = ord_P(f) - ord_P(g)$ we can extend the definition to the entire
quotient ring $\bar{k}(C)$
$$ ord_P: \bar{k}(C) \rightarrow \mathbb{Z}\cup \{\infty \} $$
\end{mydef}

The definition of order agrees with the one found in complex analysis.
If $ord_P(f) < 0$ f has a pole at $P$ and we write $f(P)=\infty$. 
If $ord_P(f) \ge 0$ f has a zero and is defined at $P$, so $f(P)$ can be calculated. 

\begin{prop}
 Let $C$ be a smooth curve. If $f\in \bar{k}(C)$ is not the constant function, then
$f$ has finitely many poles and zeros.
\label{prop:1}
\end{prop}
\begin{proof}
 FIXME. Prop 1.2 AEC.
\end{proof}

\begin{mydef}
 The \emph{divisor group} of a curve $C$ is the free abelian group generated by
points of $C$, denoted $Div(C)$. A divisor $D\in Div(C)$ is of the form
$$ D = \sum_{P\in C} n_P(P)$$
with $n_P\in\mathbb{Z}$ and $n_P = 0$ for almost all $P$.
\end{mydef}

With this in mind we can define the degree of a divisor as the sum of its 
coefficients. We also define the sum of a divisor as the sum in the group $E(\bar{k})$, so
$$ deg(D) = deg(\sum_{P\in C} n_P(P)) = \sum_{P\in C} n_P \in \mathbb{Z}$$
$$ sum(D) = sum(\sum_{P\in C} n_P(P)) = \sum_{P\in C} n_P P \in E(\bar{k})$$

These functions enable us to define the subgroup of divisors of degree zero,
$Div^0(C) \subset Div(C)$, so $Div^0(C) = \{ D\in Div(C) : deg(D) = 0 \}$.

Now let $C$ be a smooth curve and $f\in \bar{k}(C)$ a non-zero function. Since $f$
has finitely many poles and zeros (Prop. \ref{prop:1}) we can define the divisor of a
function as
$$ div(f) = \sum_{P\in C} ord_P(f)(P) $$
Note that $ord_P$ is a valution we have $ord_P(fg) = ord_P(f)+ord_P(g)$
for non-zero $f,g\in \bar{k}(C)$. Thus we get a group homomorphism
$$ div: \bar{k}(C)^* \rightarrow Div(C)$$

\begin{mydef}
 The \emph{principal divisors} of $C$ are the divisors of the form
$ D = div(f) $ for some non-zero $f\in \bar{k}(C)$. This is exactly
the image of the function $div$ and we denote this set by $Prin(C)$.
Note that since divisors of rational functions have the same number of poles
and zeros (when counted correctly), we have $deg(div(f)) = 0$. [EGET TEOREM?]
\end{mydef}

Two divisors are said to be \emph{equivalent} denoted $D_1 \sim D_2$ if
their difference is a principal divisor, $D_1 - D_2 = div(f)$ for some $f$. In
addition we can put a partial ordering on $Div(C)$, saying that a divisor $D$ is
\emph{positive} $\sum n_P(P)=D \geq 0$ if $n_P \geq 0$ for every $P\in C$. Furthermore
we write $D_1 \geq D_2$ to indicate that $D_1 - D_2$ is positive.

\begin{thm}
 $$ sum: Pic^0(C) \rightarrow E(\bar{k}) $$
is a group isomorphism
\end{thm}

\section{Weil pairing and Tate module}

\begin{prop}
 The \emph{multiplication by $n$} map
$$ [n] : E \rightarrow E $$
$$ P \mapsto nP $$
has order $n^2$.
\end{prop}
\begin{proof}
 Dual isogeny.
\end{proof}

A subgroup of $E(k)$ that will be of special interest to us is the group of points $P$
with finite order $n$, this is by definition the kernel of the multiplication by $n$ map.
\begin{mydef}
 The $n$-torsion subgroup denoted $E[n]$ is the group of points of order $n$ in $E$.
$$ E[n] = \{ P\in E : nP = O \} $$
\end{mydef}

We are now ready to construct a bilinear pairing between the $n$-torsion subgroups of
an elliptic curve and the roots of unity $\mu_n$. This will prove useful to us in coming
proofs. In addition it has well established applications within number theory, cryptography
and indentity based encryption.

The pairing we want to construct is of the form
$$ e_n : E[n] \times E[n] \rightarrow \mu_n $$
Let $T\in E[n]$ be an $n$-torsion point. From [et teorem] we know that there exists
$f \in \bar{k}(E)$ such that $div(f) = n(T) - n(O)$. Now letting $T' \in E[n^2]$ be such
that $nT' = T$, we have a function $g \in \bar{k}(E)$ such that
$$ div(g) = \sum_{R\in E[n]} (T'+R)-(R) $$
This follows from the fact that there are $n^2$ points in $E[n]$, the points $(R)$ in the
sum cancel, so we are left with $n^2 T' = nT = O$. Clearly $deg(div(g)) = 0$.

If we now form the composition $f \circ [n]$, we notice that the points $P = T' + R$ with
$R\in E[n]$ are those with the property $nP = T$. Now since $f$ has a root at $T$ from
construction, we see that $f \circ [n]$ has a root at $P$. Using the fact that $ord_P$ is a valuation
so that $div(g^n) = n\;div(g)$, and writing out the divisors of our functions we see that
$$ div(f \circ [n]) = n\sum_{R\in E[n]} (T'+R) - n\sum_{R\in E[n]} (R) = div(g^n) $$

Since our two rational functions $f \circ [n]$ and $g^n$ have the same divisors, they have the
same poles and zeros. Therefor they differ by multiplication of a constant, so
$f \circ [n] = \lambda g^n$ with $\lambda \in \bar{k}$. With a suitable choice of $\lambda$
we can assume that 
$$f \circ [n] = g^n$$
Letting $S \in E[n]$ be another $n$-torsion point and $X \in E(\bar{k})$ a point on the curve we calculate that
$$ g(X + S)^n = (f \circ [n])(X + S) = f([n]X + [n]S) = f([n]X) = g(X)^n $$
\begin{mydef}
 Given the above calculation the \emph{Weil pairing} is defined as
$$ e_n : E[n] \times E[n] \rightarrow \mu_n$$
$$ (S,T) \mapsto \frac{g(X + S)}{g(X)} $$
\end{mydef}

\begin{prop}
We have the following isomorphism of abelian groups
 $$ E[n] \simeq \mathbb{Z}/m\mathbb{Z} \times \mathbb{Z}/m\mathbb{Z} $$
\end{prop}
\begin{proof}
 fund. teorem. osv.
\end{proof}

The last proposition enables us to view automorphism of $E[n]$ as $2\times 2$ invertible matrices,
so we obtain a  mod $\ell$ galois representation
$$ Gal(\bar{k}/k) \overset{\rho}{\rightarrow} Aut(E[n]) \simeq GL_2(\mathbb{Z}/m\mathbb{Z}) $$
To avoid working with congruences and instead work with equalities, we can construct
and work with a field of characteristic 0. This is done by taking the inverse limit of the sequence
$$ \dots \overset{[l]}{\rightarrow} E[\ell^{n+1}] \overset{[l]}{\rightarrow} E[\ell^{n}] \overset{[l]}{\rightarrow} E[\ell^{n-1}] \rightarrow \ldots $$
$$ T_\ell(E) = \varprojlim E[\ell^n] $$
This is called the \emph{$\ell$-adic Tate module} of E. Notice that since each the groups $E[\ell^n]$ has a
$\mathbb{Z}/\ell^n\mathbb{Z}$-module structure, $T_\ell(E)$ will have natrual structure as a module
over the ring og $\ell$-adic integers $\mathbb{Z}_\ell$.

Similarly we can in a sense ``glue'' together the Weil pairings
$$ e_{\ell^n} : E[\ell^n] \times E[\ell^n] \rightarrow \mu_{\ell^n} $$
by constructing the $\ell$-adic roots of unity, and we obtain what is called the
\emph{$\ell$-adic Weil pairing}.
$$ e: T_\ell(E) \times T_\ell(E) \rightarrow T_\ell(\mu) $$

\section{Frobnius and finite fields}
Throughout this section our fields $k$ will be finite, so let $char(k) = p$ for
a prime $p$. This means that $k = \mathbb{F}_{q}$ for some $q = p^r$.

\begin{mydef}
 The \emph{frobenius} endomorphism is the $q^{th}$-power map
$$ \phi: k \rightarrow k $$
$$ x \mapsto x^q $$
which induces a map on curves as follows
$$ \phi: E(k) \rightarrow E(k) $$
$$ (x_0,\ldots , x_n) \mapsto (x_0^q, \ldots , x_n^q) $$
\end{mydef}

\begin{prop}
 The degree map
$$ deg: Hom(E_1, E_2) \rightarrow \mathbb{Z} $$
is a positive quadratic form.
\end{prop}
\begin{proof}
 Clearly $deg(f) = deg(-f)$. The only thing that takes a proof is the
bilinearity of the pairing
$$ End(E_1, E_2) \times End(E_1, E_2) \rightarrow \mathbb{Z}$$
$$ (\phi, \psi) \mapsto deg(\phi + \psi) - deg(\phi) - deg(\psi) $$
For this proof we will make extentive use of the dual isogeny, but first
notice that we have an injection of multiplication by $n$ maps:
$$ [\quad]: \mathbb{Z} \rightarrow End(E_1) $$
A calculation then yields 
\begin{eqnarray*} 
 [<\phi,\psi>] &=& [deg(\phi+\psi)]-[deg(\phi)]-[deg(\psi)] \nonumber \\
               &=& (\hat{\phi+\psi})(\phi+\psi) - \hat{\phi}\phi - \hat{\psi}\psi \nonumber \\
	       &=& \hat{\phi}\psi + \hat{\psi}\phi
\end{eqnarray*}
The pairing is then shown to be linear in the first varible, the second variable is
similar.
\begin{eqnarray*}
 [<\phi_1+\phi_2, \psi>] &=& \hat{\psi}(\phi_1+\phi_2) + (\hat{\phi_1+\phi_2})\psi \nonumber \\
			 &=& (\hat{\psi}\phi_1+\hat{\phi_1}\psi) + (\hat{\psi}\phi_2 + \hat{\phi_2}\psi) \nonumber \\
			 &=& [<\phi_1,\psi>] + [<\phi_2,\psi>] 
\end{eqnarray*}
\end{proof}

\begin{thm}
 Let $\phi$ be the $q^{th}$ frobenius map. Then the map $1-\phi$ is seperable, and
$\#ker(1-\phi) = deg(1-\phi)$.
\end{thm}
\begin{proof}
 Proofs by the means of galois theory are given in [silverman-referanse], more
elementary proofs are available in [lawrence-ref].
\end{proof}

\begin{lemma}
 \textbf{(Cauchy-Schwartz inequality)}. Let $A$ be an abelian group and
$$ d: A \rightarrow \mathbb{Z} $$
a positive definite quadratic form. Then for all $\psi, \phi \in A$ the following holds
$$ |d(\psi-\phi)-d(\phi)-d(\psi)| \leq 2 \sqrt{d(\phi)d(\psi)} $$
\end{lemma}
\begin{proof}
 fixme
\end{proof}



\section{Schoof's algorithm}


\end{document}
