\documentclass[a4paper,11pt,latin1]{amsart}
\usepackage{lmodern}
\usepackage{amsthm,mathrsfs,amsmath}
\usepackage[all,cmtip]{xy}
\usepackage{listings}
\usepackage{algorithmic} %pseudokode
\usepackage{graphicx}

%opening
%\title{$p$-adic and $\ell$-adic point counting on elliptic curves}
\title{Counting Points on Elliptic Curves}
\author{Ole Andr\'{e} Birkedal}

\begin{document}
\newtheorem{thm}{Theorem}
\newtheorem{mydef}{Definition}
\newtheorem{ex}{Example}
\newtheorem{prop}{Proposition}
\newtheorem{lemma}{Lemma}
\newtheorem{cor}{Corollary}

\begin{abstract}
Elliptic curve cryptography is an approach to public-key cryptography.
The elliptic curve chosen should be such that its number of points is divisible by
a large prime factor. Ideally there should be a prime number of points on the curve.

In this paper we present the first efficient point counting algorithm due to Schoof,
before giving a significant improvement due to Elkies. In the final section I give Satoh's
algorithm which is even faster for fields of small
characteristic, and has paved the way for the field of $p$-adic point counting.
\end{abstract}

\maketitle
\newpage
\section*{Preface}
At the start of my final year at the university I was fairly uncertain as to what my thesis would
be about. After having consulted several potential advisors, all with
very interesting topics, I finally talked to Kristian Gj\o steen
who presented me with suggestion for a thesis concerning elliptic curves.
I had for a long time been interested in number theory which is why I finally settled on this topic.

I would like to thank my high school teacher Thuy Skj\ae veland for being the first to spark
my interest in mathematics. In addition I would like to thank Erlend Hamberg for all the
much needed coffee and backgammon breaks through the semester.

A special thanks goes out to my thesis advisor Kristian Gj\o steen who always explained
things in such a way that even I could grasp them. This guy has an answer to everything.
\newpage
\tableofcontents



\section{Algebraic geometry}
In this section we define the fundamental objects in algebraic geometry and state
some facts about their structure. We will then move on to the theory of
curves and Weil divisors.

\begin{mydef}
\emph{Projective n-space} over a field $k$ denoted $\mathbb{P}^n$ is the set 
of all $(n+1)$-tuples $$(x_0,\ldots,x_n)\in\mathbb{A}^{n+1}$$
modulo the equivalence relation given by $(x_0,\ldots,x_n)\sim(y_0,\ldots,y_n)$ 
if there exists $\lambda\in k$ such that $x_i=\lambda y_i$.
The equivalence class $\{(x_0,\ldots,x_n)\}$ is denoted $[x_0,\ldots,x_n]$.
Here $\mathbb{A}^n = \{ (x_1,\ldots,x_n) : x_i \in \bar{k} \} $ is the affine $n$-space.
\end{mydef}

Let $Gal(\bar{k}/k)$ be the galois group of $\bar{k}/k$. This group acts on
$\mathbb{A}^n$, such that when $\sigma \in Gal(\bar{k}/k)$ and $P\in \mathbb{A}^n$
we define $\sigma(P) = (\sigma(x_1),\ldots,\sigma(x_n))$. Now we define
the set of $k$-rational points in $\mathbb{A}^n$ to be those fixed under action by
the galois group
$$ \mathbb{A}^n(k) = \{ P \in \mathbb{A}^n : \sigma(P) = P\, \forall\, \sigma \in
Gal(\bar{k}/k) \} $$

Similarly we define the set of $k$-rational points in $\mathbb{P}^n$ to be
$$ \mathbb{P}^n(k) = \{ P \in \mathbb{P}^n : \sigma(P) = P\, \forall\, \sigma \in 
Gal(\bar{k}/k) \} $$

\begin{mydef}
 A polynomial $f\in\bar{k}[X]$ is said to be \emph{homogeneous of degree $d$} if for all
$\lambda\in\bar{k}$ we have.
$$f(\lambda x_0,\ldots,\lambda x_n) = \lambda^d f(x_0,\ldots,x_n)$$
Furthermore an ideal $I\subseteq\bar{k}[X]$ is said to be homogeneous if it is generated
by homogeneous polynomials.
\end{mydef}

\begin{mydef}
 A \emph{projective algebraic set} is of the form
$$ V_I = \{ P\in \mathbb{P}^n : f(P) = 0\, \forall homogeneous\, f\in I \} $$
Given such a set $V$ we associate to it an ideal $I(V) \in \bar{k}[X]$ generated by
$$ \{f\in\bar{k} : f\, homogeneous\, and\, f(P)=0\, \forall P\in V \} $$
\end{mydef}

\begin{mydef}
 A projective algebraic set is called a \emph{projective variety} if the homogeneous
ideal defined above is a prime ideal in $\bar{k}[x]$.
\end{mydef}

\begin{mydef}
 Let $V/k$ be a projective variety (i.e. V defined over $k$), then the projective coordinate
ring of $V/k$ is defined by
$$ k[V] = \frac{k[x]}{I(V/k)}$$
Note that since $I(V/k)$ is a prime ideal, the coordinate ring is an integral domain.
This enables us to form the quotient field of $k[V]$ which we denote $k(V)$, and it is called
the \emph{function field} of $V$.
\end{mydef}

A rather interesting ideal to keep in mind is given by
$$ M_p = \{ f\in \bar{k}[V] : f(P)=0 \} $$
This is a maximal ideal because the map $\phi: \bar{k}[V] \rightarrow \bar{k}$ given by
$ f \mapsto f(P) $ has kernel exactly $M_p$. It is clearly onto, so it induces an
isomorphism $$\tilde{\phi}: \bar{k}[V]/M_p \rightarrow \bar{k} $$

\begin{mydef}
 The \emph{localization of $\bar{k}[V]$ at $M_p$} is given by
$$ \bar{k}[V]_P = \{ h \in \bar{k}[V] : h = f/g\, f,g\in \bar{k}[V]\, and\, g(P)=0 \} $$
The functions in $\bar{k}[V]_P$ are all defined at $P$.
\end{mydef}

\begin{ex}
 If $V$ is a variety given by a single non-constant polynomial equation
$$f(x_1,\ldots,x_n) = 0$$ 
then the dimension of the variety $dim(V)$ is $n-1$. The (projective) varieties
we will study are called \emph{elliptic curvevs} and are
given by polynomial equations
$$E: y^2 = x^3+ax+b$$
They correspond to polynomials of the form $f(x,y) = x^3+ax+b-y^2$ so $dim(E)=1$.
We say curves are projective varieties of dimension $1$.
\end{ex}

The objects we will be working on are projective varieties, but they are not
very interesting unless we define maps between them.

\begin{mydef}
 Let $V_1$ and $V_2$ be projective varieties, a \emph{rational map} $\phi: V_1 \rightarrow V_2$
is a set of maps $\{\phi_0,\ldots,\phi_n\}$ with $\phi_i \in \bar{k}(V_1)$ such that for every
$P\in V_1$ we define
$$\phi(P) = [\phi_0(P),\ldots,\phi_n(P)] \in V_2$$
Such a rational map is called a \emph{morphism} if it is defined at every point $P$.
\end{mydef}

The varieties and the morphisms between them make up a category, so our next
definition of an isomorphism will be the general one found in category theory.

\begin{mydef}
 Two varieties $V$ and $W$ are \emph{isomorphic} denoted $V\simeq W$
if there exist morphisms $\phi: V \rightarrow W$ and $\psi: W \rightarrow V$ such that
$\phi \psi = 1_W$ and $\psi \phi = 1_V$.
If the rational functions $\psi$ and $\phi$ are defined over $k$ we say that $V$ and $W$
are isomorphic over $k$. If not, they are isomorphic over some field extension of $k$
(i.e. $\bar{k}$).
\end{mydef}


\subsection{Curves and divisors}
Recall that curves are projective varieties of dimension one. Even more special
are elliptic curves, which are curves with \emph{genus} equal to 1. This will
be introduced later on. These are in practise the only curves we will be working with.

\begin{mydef}
 Let $C$ be a curve and $P\in C$ a non-singular point on the curve. A valuation on
$\bar{k}[C]_P$ is given by
$$ ord_P : \bar{k}[C]_P \rightarrow \{ 0, 1, 2, \ldots \} \cup \{ \infty \} $$
$$ ord_P(f) = max \{ d\in \mathbb{Z} : f\in M_P^d \} $$
This is called \emph{the order of $f$ at $P$}.
Letting $ord_P(f/g) = ord_P(f) - ord_P(g)$ we can extend the definition to the entire
quotient ring $\bar{k}(C)$
$$ ord_P: \bar{k}(C) \rightarrow \mathbb{Z}\cup \{\infty \} $$
\end{mydef}

The definition of order agrees with the one found in complex analysis.
If $ord_P(f) < 0$ f has a pole at $P$ and we write $f(P)=\infty$. 
If $ord_P(f) \ge 0$ f has a zero and is defined at $P$, so $f(P)$ can be calculated. 

\begin{prop}
 Let $C$ be a smooth curve. If $f\in \bar{k}(C)$ is not the constant function, then
$f$ has finitely many poles and zeros.
\label{prop:1}
\end{prop}
\begin{proof}
 FIXME. Prop 1.2 AEC.
\end{proof}

\begin{mydef}
 The \emph{divisor group} of a curve $C$ is the free abelian group generated by
points of $C$, denoted $Div(C)$. A divisor $D\in Div(C)$ is of the form
$$ D = \sum_{P\in C} n_P(P)$$
with $n_P\in\mathbb{Z}$ and $n_P = 0$ for almost all $P$.
\end{mydef}

With this in mind we can define the degree of a divisor as the sum of its 
coefficients. We also define the sum of a divisor as the sum in the group $E(\bar{k})$, so
$$ deg(D) = deg(\sum_{P\in C} n_P(P)) = \sum_{P\in C} n_P \in \mathbb{Z}$$
$$ sum(D) = sum(\sum_{P\in C} n_P(P)) = \sum_{P\in C} n_P P \in E(\bar{k})$$

These functions enable us to define the subgroup of divisors of degree zero,
$Div^0(C) \subset Div(C)$, so $Div^0(C) = \{ D\in Div(C) : deg(D) = 0 \}$.

Now let $C$ be a smooth curve and $f\in \bar{k}(C)$ a non-zero function. Since $f$
has finitely many poles and zeros (Prop. \ref{prop:1}) we can define the divisor of a
function as
$$ div(f) = \sum_{P\in C} ord_P(f)(P) $$
Note that $ord_P$ is a valution we have $ord_P(fg) = ord_P(f)+ord_P(g)$
for non-zero $f,g\in \bar{k}(C)$. Thus we get a group homomorphism
$$ div: \bar{k}(C)^* \rightarrow Div(C)$$

\begin{mydef}
 The \emph{principal divisors} of $C$ are the divisors of the form
$ D = div(f) $ for some non-zero $f\in \bar{k}(C)$. This is exactly
the image of the function $div$ and we denote this set by $Prin(C)$.
Note that since divisors of rational functions have the same number of poles
and zeros (when counted correctly), we have $deg(div(f)) = 0$. [EGET TEOREM?]
\end{mydef}

Two divisors are said to be \emph{equivalent} denoted $D_1 \sim D_2$ if
their difference is a principal divisor, $D_1 - D_2 = div(f)$ for some $f$. In
addition we can put a partial ordering on $Div(C)$, saying that a divisor $D$ is
\emph{positive} $\sum n_P(P)=D \geq 0$ if $n_P \geq 0$ for every $P\in C$. Furthermore
we write $D_1 \geq D_2$ to indicate that $D_1 - D_2$ is positive.

\begin{ex}
 Inequalities can easily summarize some key properties of a function. So instead of
saying $f \in \bar{k}(C)$ is regular everywhere except at $P$ and $Q$, where it has a
pole and a root of order $m$ and $n$ respectively, we could write
$$ div(f) \geq -m(P)+n(Q) $$
\end{ex}

The last example motivates our next definition, where we collect all functions
which satisy some inequality. This turns out to make up a finite dimensional
$\bar{k}$-vector space.

\begin{mydef}
 Let $D \in Div(C)$ be a divisor, and we define the set of functions
$$ \mathscr{L}(D) = \{ f\in \bar{k}(C) : div(f) \geq -D \} \cup \{ 0\} $$
\end{mydef}

\begin{prop}
 $\mathscr{L}(D)$ is a finite dimensional $\bar{k}$-vector space, and we denote
its dimension by
$$ \ell(D) = dim_{\bar{k}} \mathscr{L}(D) $$
\end{prop}
\begin{proof}
 First note that if $D' > D$ then $D' = D + P_1 \ldots P_s$, so we get an ascending chain of
subspaces
$$ \mathscr{L}(D) \subseteq \mathscr{L}(D+P_1) \subseteq \ldots \subseteq \mathscr{L}(D+P_1 \ldots P_s) $$

\end{proof}

\begin{mydef}
 The space of differential forms on a curve $C$ is a $\bar{k}(C)$-vector space denoted $\Omega_C$
generated by symbols subject to the releations known from analysis. 
For $x, y \in \bar{k}(C)$ and $a \in \bar{k}$
\begin{enumerate}
  \item d(x+y) = dx + dy
  \item d(xy) = xdy + ydx
  \item da = 0
\end{enumerate}
Let $f_i \in \bar{k}(C)$ and $dx_i$ be the symbols as defined above, a general
element $\omega \in \Omega_C$ is of the form
$$ \omega = \sum f_i dx_i $$
\end{mydef}

Divisors in the image of the map $div: \Omega_C \rightarrow Pic(C)$ are called
\emph{canonical divisors}. They will play a role in the next theorem which will
serve as an important tool for calculating the dimension of the vector space
$\mathscr{L}(D)$, which will be crucial in establishing an important isomorphism.
It will also serve as a definition of the genus $g$.

\begin{thm}
 \textbf{(Riemann-Roch)}
  Let $C$ be a smooth curve and $K_C$ a canonical divisor on $C$. Then for
any $D \in Div(C)$ we have
$$ \ell(D) - \ell(K_C - D) = deg(D) - g + 1 $$
where $g \geq 0$ is called the \emph{genus} of the curve $C$.
\end{thm}
\begin{proof}
 A proof would be outside the scope of this paper + referanser.
\end{proof}

Given a non-constant map of curves $\phi: C_1 \rightarrow C_2$, we have an induced map
on function fields $\phi^*: K(C_2) \rightarrow K(C_1)$. From this again we get an induced
map on differential forms
$$ \phi^*: \Omega_{C_2} \rightarrow \Omega_{C_1} $$
$$ \phi^*\left(\sum f_i dx_i\right) = \sum (\phi^* f_i) d(\phi^* x_i) $$

\begin{mydef}
 Let $\phi: C_1 \rightarrow C_2$ be a map of curves and $\phi^*$ its induced map on
function fields. We then say that $\phi$ is a \emph{seperable map} if
$K(C_1)/\phi^* K(C_2)$ is a seperable extension.
\end{mydef}

Recall that a field extension is separable if and only if the derivative of the minimal
polynomial for each element is non-zero. This fact is the motivation for our next result,
which gives a useful criterion for determining when a map is separable.

\begin{prop}
 Let $\phi: C_1 \rightarrow C_2$ be a map of curves, then $\phi$ is separable if and only if
the induced map $\phi^*: \Omega_{C_2} \rightarrow \Omega_{C_1} $ is non-zero.
\end{prop}

\begin{thm}
 $$ sum: Pic^0(C) \rightarrow E(\bar{k}) $$
is a group isomorphism
\end{thm}
\begin{proof}
 We begin by showing that there is a unique point $P \in E(\bar{k})$ associated to
each $D \in Div^0(E)$ as follows
$$ D \sim (P) -(O) $$
This will be given by a map
$$ \sigma: Div^0(E) \rightarrow E(\bar{k}) $$
From [referanse] we have that $\ell(D+(O)) = deg(D+(O)) = 1$ since $deg(D) = 0$.
Let then $f \in \bar{k}(E)$ be a generator for $\mathscr{L}(D+(O))$, so by definition
$$ div(f) \geq -D-(O) $$
But since $deg(div(f)) = 0$ and $deg(-D-(O)) = -1$ we have for some $P \in E(\bar{k})$ that
$$ div(f) = -D-(O)+(P) $$
which is exactly the definition of
$$ D \sim (P) - (O) $$

This point $P$ is unique, because if we assume that $P'$ is another point with the same
property, then
$$ (P) \sim D + (O) \sim (P') $$
so by [3.3 korrolar til riemann-roch]
$ P = P'$.

The map $\sigma$ is easily seen to be a surjection, because for any $P \in E(\bar{k})$ we have
$$ \sigma((P)-(O)) = P $$

Now if we can show that the kernel of $\sigma$ is exactly the principal divisors we are done.
Let us assume that $\sigma(D) = O$ so from definition we have that $D \sim (O)-(O) \sim (O)$
meaning $D - (O) = div(f)$ for some $f \in \bar{k}(E)$, so $D = div(f)$ is principal.
For the other implication we assume that $D = div(f)$ is principal. Using the definition and
letting $P$ be any point and $f, f' \in \bar{k}(E)$ a calculation yields
$$ \sigma(D) = \sigma(div(f)) = (P)-(O) $$
$$ div(f) \sim (P) - (O) $$
$$ div(f) - (P) - (O) = div(f') $$
$$ (O) - (P) = div(f') - div(f) = div(f' f) $$
So $ (P) \sim (O) $ which implies $P = O$ from [3.3-korrolaret].

We have thus established the group isomorphism which we again will call $\sigma$
$$ \sigma : Div^0(E)/Prin(E) = Pic^0(E) \rightarrow E(\bar{k}) $$

\end{proof}

\section{P-adic numbers}
Working with a field of characteristic $0$ is + motivasjon.


\section{Weil pairing and Tate module}

\begin{prop}
 The \emph{multiplication by $n$} map
$$ [n] : E \rightarrow E $$
$$ P \mapsto nP $$
has order $n^2$.
\end{prop}
\begin{proof}
The shortest proof relies heavily on the dual isogeny, so letting $d = \deg [n]$ and using the properties of 
the dual isogeny we calculate
$$ [d] = \widehat{[n]}[n] = [n][n] = [n^2] $$
and since $End(E)$ is torsion free \cite{AEC} we get that $d = n^2$.
\end{proof}

A subgroup of $E(k)$ that will be of special interest to us is the group of points $P$
with finite order $n$, this is by definition the kernel of the multiplication by $n$ map.
\begin{mydef}
 The $n$-torsion subgroup denoted $E[n]$ is the group of points of order $n$ in $E$.
$$ E[n] = \left\{ P\in E : nP = O \right\}.$$
\end{mydef}

We are now ready to construct a bilinear pairing between the $n$-torsion subgroups of
an elliptic curve and the roots of unity $\mu_n$. This will prove useful to us in coming
proofs. In addition it has well established applications within number theory, cryptography
and identity based encryption.

The pairing we want to construct is of the form
$$ e_n : E[n] \times E[n] \rightarrow \mu_n.$$
Let $T\in E[n]$ be an $n$-torsion point. From \cite{Lawrence} we know that there exists
$f \in \bar{k}(E)$ such that $div(f) = n(T) - n(O)$. Now letting $T' \in E[n^2]$ be such
that $nT' = T$, we have a function $g \in \bar{k}(E)$ such that
$$ div(g) = \sum_{R\in E[n]} (T'+R)-(R). $$
This follows from the fact that there are $n^2$ points in $E[n]$, the points $(R)$ in the
sum cancel, so we are left with $n^2 T' = nT = O$. Clearly $\deg(div(g)) = 0$.

If we now form the composition $f \circ [n]$, we notice that the points $P = T' + R$ with
$R\in E[n]$ are those with the property $nP = T$. Now since $f$ has a root at $T$ from
construction, we see that $f \circ [n]$ has a root at $P$. Using the fact that $ord_P$ is a valuation
so that $div(g^n) = n\;div(g)$, and writing out the divisors of our functions we see that
$$ div(f \circ [n]) = n\sum_{R\in E[n]} (T'+R) - n\sum_{R\in E[n]} (R) = div(g^n).$$

Since our two rational functions $f \circ [n]$ and $g^n$ have the same divisors, they have the
same poles and zeros. Therefore they differ by multiplication of a constant, so
$f \circ [n] = \lambda g^n$ with $\lambda \in \bar{k}$. With a suitable choice of $\lambda$
we can assume that 
$$f \circ [n] = g^n.$$
Letting $S \in E[n]$ be another $n$-torsion point and $X \in E(\bar{k})$ a point on the curve we calculate that
$$ g(X + S)^n = (f \circ [n])(X + S) = f([n]X + [n]S) = f([n]X) = g(X)^n. $$
\begin{mydef}
 Given the above calculation the \emph{Weil pairing} is defined as
$$ e_n : E[n] \times E[n] \rightarrow \mu_n$$
$$ (S,T) \mapsto \frac{g(X + S)}{g(X)}. $$
\end{mydef}

\begin{prop}
 The Weil pairing $e_n$ satisfies the following properties
\begin{enumerate}
 \item Bilinear in both variables: $e_n(P_1 + P_2, Q) = e_n(P_1,Q)e_n(P_2,Q)$ and similarly for the other variable.
 \item Alternating: $e_n(P,Q) = e_n(Q,P)^{-1}$.
 \item Non-degenerate: If $e_n(P,Q) = 1$ for all $P \in E[n]$ then $Q=O$.
 \item Galois invariant: For all $\sigma \in Gal(\bar{k}/k)$ we have $e_n(P,Q)^\sigma = e_n(\sigma(P),\sigma(Q))$.
\end{enumerate}
\end{prop}
\begin{proof}
 See \cite{AEC}.
\end{proof}

\begin{prop}
Letting $\ell$ be a prime not dividing $char(k)$ we have the following isomorphism of abelian groups
 $$ E[\ell] \simeq \mathbb{Z}/\ell\mathbb{Z} \times \mathbb{Z}/\ell\mathbb{Z}. $$
\end{prop}

The last proposition enables us to view automorphism of $E[\ell]$ as $2\times 2$ invertible matrices,
so we obtain a  mod $\ell$ Galois representation
$$ Gal(\bar{k}/k) \overset{\rho}{\rightarrow} Aut(E[\ell]) \simeq GL_2(\mathbb{Z}/\ell\mathbb{Z}). $$
To avoid working with congruences and instead work with equalities, we can construct
and work with a field of characteristic 0. This is done by taking the inverse limit 
as introduced in Chapter \ref{p-adic} of the sequence
$$ \dots \overset{[l]}{\rightarrow} E[\ell^{n+1}] \overset{[l]}{\rightarrow} E[\ell^{n}] \overset{[l]}{\rightarrow} E[\ell^{n-1}] \rightarrow \ldots $$
$$ T_\ell(E) = \varprojlim E[\ell^n]. $$
This is called the \emph{$\ell$-adic Tate module} of E. Notice that since each of the groups $E[\ell^n]$ has a
$\mathbb{Z}/\ell^n\mathbb{Z}$-module structure, $T_\ell(E)$ will have natural structure as a module
over the ring of $\ell$-adic integers $\mathbb{Z}_\ell$ as follows: $r\in\mathbb{Z}_\ell$ and
$x=(x_n)_n\in T_\ell(E)$ then
$$ r(x_n)_n = ((r \mod p^n)x_n)_n.$$

Similarly we can in a sense ``glue'' together the Weil pairings
$$ e_{\ell^n} : E[\ell^n] \times E[\ell^n] \rightarrow \mu_{\ell^n} $$
by constructing the $\ell$-adic roots of unity, and we obtain what is called the
\emph{$\ell$-adic Weil pairing}
$$ e: T_\ell(E) \times T_\ell(E) \rightarrow T_\ell(\mu). $$

We end this section by stating a result which enables us to view endomorphisms on $E$ as
$2\times 2$ matrices by choosing a $\mathbb{Z}_\ell$ basis.
\begin{prop}
  The map $\phi: End(E) \rightarrow End(T_\ell(E))$ is an injective ring homomorphism. In addition
we have that $End(T_\ell(E)) \simeq M_2(\mathbb{Z}_\ell)$ where $M_2$ is the ring of $2\times 2$
matrices with $\mathbb{Z}_\ell$ coefficients.
\end{prop}
This will prove useful later on when establishing relations between determinants,
trace and degrees of maps.
\section{Frobnius and finite fields} \label{frob}
Throughout this section our fields $k$ will be finite, so let $char(k) = p$ for
a prime $p$. This means that $k = \mathbb{F}_{q}$ for some $q = p^r$.

\begin{mydef}
 The \emph{Frobenius} endomorphism is the $p^{th}$-power map
$$ \phi: k \rightarrow k $$
$$ x \mapsto x^p $$
which induces a map on curves as follows
$$ \phi: E(k) \rightarrow E^{\phi}(k) $$
$$ (x_0,\ldots , x_n) \mapsto (x_0^p, \ldots , x_n^p) $$
where $E^{\phi}$ is the curve $E$ with $\phi$ applied to its coefficients.
$$E: y^2 = x^3 + ax + b \quad E: y^2 = x^3 + \phi(a)x + \phi(b) $$
\end{mydef}

We can apply the Frobenius endomorphism $r$ times $$\phi^r(x) = x^{p^r} = x^q$$
And since every finite field of $q$ elements is the splitting field of $x^{q}-x$, it is in other words
the fixed points of the $q^{th}$ Frobenius endomorphism
$$ \phi^r(x) = x \iff x \in \mathbb{F}_q $$
The same is true for all intermediate fields of size $p^k$ with $0 < k \leq r$, so in general
we have that the $\phi^k$ fixes the elements of the field $\mathbb{F}_{p^k}$.
\begin{prop}
 The degree map
$$ deg: Hom(E_1, E_2) \rightarrow \mathbb{Z} $$
is a positive quadratic form.
\end{prop}
\begin{proof}
 Clearly $deg(f) = deg(-f)$. The only thing that takes a proof is the
bilinearity of the pairing
$$ End(E_1, E_2) \times End(E_1, E_2) \rightarrow \mathbb{Z}$$
$$ (\phi, \psi) \mapsto deg(\phi + \psi) - deg(\phi) - deg(\psi) $$
For this proof we will make extentive use of the dual isogeny, but first
notice that we have an injection of multiplication by $n$ maps:
$$ [\quad]: \mathbb{Z} \rightarrow End(E_1) $$
A calculation then yields 
\begin{eqnarray*} 
 [\langle \phi,\psi \rangle] &=& [deg(\phi+\psi)]-[deg(\phi)]-[deg(\psi)] \nonumber \\
               &=& (\widehat{\phi+\psi})(\phi+\psi) - \widehat{\phi}\phi - \widehat{\psi}\psi \nonumber \\
	       &=& \widehat{\phi}\psi + \widehat{\psi}\phi
\end{eqnarray*}
The pairing is then shown to be linear in the first varible, the second variable is
similar.
\begin{eqnarray*}
 [\langle \phi_1+\phi_2, \psi \rangle] &=& \widehat{\psi}(\phi_1+\phi_2) + (\widehat{\phi_1+\phi_2})\psi \nonumber \\
			 &=& (\widehat{\psi}\phi_1+\widehat{\phi_1}\psi) + (\widehat{\psi}\phi_2 + \widehat{\phi_2}\psi) \nonumber \\
			 &=& [\langle \phi_1,\psi \rangle] + [\rangle \phi_2,\psi \rangle] 
\end{eqnarray*}
\end{proof}

\begin{thm} \label{frobkernel}
 Let $\phi$ be the $q^{th}$ frobenius map on $E/\mathbb{F}_q$. Then the map $1-\phi$ is seperable, and
$\#ker(1-\phi) = deg(1-\phi)$.
\end{thm}
\begin{proof}
  Recall from chapter \ref{diffsep} that a map $\psi$ is separable if and only if $\psi^*(\omega) \neq 0$,
where $\omega$ is the invariant differential. Using that the Frobenius $\phi$ is inseparable \cite{AEC}
we compute
\begin{eqnarray}
 (1-\phi)^*(\omega) &=& [1]^*\omega - \phi^*(\omega) \nonumber \\
		    &=& \omega - 0 \nonumber \\
		    &=& \omega \nonumber
\end{eqnarray}
thus $(1-\phi)^*(\omega) = 0$ if and only if $\omega = 0$, but the invariant differential is non-zero
so $(1-\phi)^*(\omega) \neq 0$ which means $1-\phi$ is separable.
\end{proof}

\begin{lemma}
 \textbf{(Cauchy-Schwartz inequality)}. Let $A$ be an abelian group and
$$ d: A \rightarrow \mathbb{Z} $$
a positive definite quadratic form. Then for all $\psi, \phi \in A$ the following holds
$$ |d(\psi-\phi)-d(\phi)-d(\psi)| \leq 2 \sqrt{d(\phi)d(\psi)} $$
\end{lemma}
\begin{proof}
 Let $\psi, \phi \in A$. From the definition of a quadratic form there is a bilinear pairing
$$ L(\psi, \phi) = d(\psi-\phi) - d(\psi) - d(\phi) $$
Using this definition, the fact that $d$ is positive definite and letting $m,n \in \mathbb{Z}$ where
$m = -L(\psi, \phi)$ and $n = 2d(\psi)$ we calculate

\begin{eqnarray}
 0 \leq d(m\psi - n\phi) &=& d(m\psi) + L(m\psi, n\phi) + d(n\phi) \nonumber \\
			 &=& m^2 d(\psi) + mnL(\psi,\phi) + n^2 d(\phi) \nonumber \\
			 &=& d(\psi) \left( 4d(\psi)d(\phi)-L(\psi, \phi)^2 \right) \nonumber 
\end{eqnarray}

where on the last line we make the substitution. If $d(\psi)=0$ the inequality is trivial, if
$d(\psi) \neq 0$ then we divide it out and obtain our result
$$L(\psi, \phi)^2 \leq 4d(\psi)d(\phi) $$
\end{proof}

\begin{thm}
 \textbf{(Hasse's theorem)}. Let $E$ be an elliptic curve over a finite field $k$ with $q$ elements, then
$$ |\#E(k) - q - 1| \leq 2\sqrt{q} $$
\end{thm}
\begin{proof}
 We let $\phi_q: E \rightarrow E$ be the $q^{th}$ Frobenius endomorphism on $E$ given by 
$(x,y) \mapsto (x^q, y^q)$. Recall that $\phi_q$ fixes our field of $q$ elements, thus
$$ P \in E(k) \quad \iff \quad \phi_q(P) = P$$
Writing out the right hand side of the implication we see that
$$ 0 = P - \phi_q(P) = (1 - \phi_q)(P) $$
which enables us to count the number of points in $E(k)$ by counting the number of points in the kernel
of the seperable map $1-\phi_q$. Recall from before that the number of points in the kernel is equal
to the degree of the seperable map
$$ \#E(k) = \# ker(1-\phi_q) = deg(1-\phi_q) $$
We have shown in that the degree map on $End(E)$ is a positive definite quadratic form, so
by using the inequality from the previous theorem we calculate
$$|deg(1-\phi_q) - deg(\phi_q) - deg(1)| = |\#E(k) - q - 1| \leq 2\sqrt{deg(\phi_q)} = 2\sqrt{q}$$
\end{proof}

\begin{prop} 
 If $\psi \in End(E)$ then $det(\psi_\ell) = deg(\psi)$, where $\psi_\ell$ is a $2\times2$ matrix acting
on the Tate module $T_\ell(E)$.
\label{detdeg}
\end{prop}
\begin{proof}
 We fix a basis $v_1,v_2 \in \mathbb{Z}_\ell \times \mathbb{Z}_\ell$ for $T_\ell(E)$ and denote the matrix
associated to this basis by
$$ \psi_\ell = \begin{pmatrix} a & b \\ c & d \end{pmatrix} $$
We now calculate by relying heavily on the $\ell$-adic Weil pairing,
$e: T_\ell(E) \times T_\ell(E) \rightarrow T_\ell(\mu)$.
\begin{eqnarray}
 e(v_1, v_2)^{deg(\psi)} &=& e([deg \psi]v_1, v_2) \nonumber \\
			 &=& e(\psi_\ell \widehat{\psi_\ell} v_1, v_2) \nonumber \\
			 &=& e(\psi_\ell v_1, \psi_\ell v_2) \nonumber \\
			 &=& e(a v_1 + c v_2, b v_1 + d v_2) \nonumber \\
			 &=& e(a v_1, d v_2) e(c v_2, b v_1) \nonumber \\
			 &=& e(a v_1, d v_2) e(b v_1, c v_2)^{-1} \nonumber \\
			 &=& e(v_1, v_2)^{ad} e(v_1, v_2)^{-bc} \nonumber \\
			 &=& e(v_1, v_2)^{ad - bc} \nonumber \\
			 &=& e(v_1, v_2)^{det \psi_\ell} \nonumber
\end{eqnarray}
Since the pairing is non-degenerate we obtain $deg(\psi) = det(\psi_\ell)$.
\end{proof}

Writing out the determinant of $1-A$ for any matrix $A$ we get
$$ \begin{vmatrix} 1-a & -b \\ -c & 1-d \end{vmatrix} = 1-(a+d)+ad-bc = 1-tr(A)+det(A) $$
so we see that $tr(\psi_\ell) = 1 + det(\psi_\ell) - det(1-\psi_\ell)$. Using the previous theorem we
get $$tr(\psi_\ell) = 1 + deg(\psi_\ell) - deg(1-\psi_\ell)$$ by substituting with the $q^{th}$ 
Frobenius endomorphism on $T_\ell(E)$ and setting $\tau = tr(\phi_q)$ we get
$$\#E(k) = 1 + q - \tau$$
where we know from Hasse's theorem that $|\tau| \leq 2\sqrt{q}$.

The next proposition will be used in chapter \ref{satoh}, it is easy to prove and gives a nice
expression of the Frobenius trace in terms of the dual isogeny.

\begin{prop}
 Let $\phi: E \rightarrow E$ be the $q^{th}$ Frobenius endomorphism and $\widehat{\phi}$ its dual, then
the following holds
$$ t = tr(\phi) = \phi + \widehat{\phi}$$
\end{prop}
\begin{proof}
 Recall that $1-\phi$ is seperable, so $$(1-\phi)(\widehat{1-\phi}) = deg(1-\phi) = \#ker(1-\phi) = \#E(k)$$
Expanding the product on the left we get
\begin{eqnarray}
 (1-\phi)(\widehat{1-\phi}) &=& (1-\phi)(1-\widehat{\phi}) \nonumber \\
			    &=& 1 - (\phi + \widehat{\phi}) + \phi\widehat{\phi} \nonumber \\
			    &=& 1 - (\phi + \widehat{\phi}) + q \nonumber
\end{eqnarray}
From before we had that $\#E(k) = q + 1 - t$ and we just calculated that $\#E(k) = q + 1 - (\phi +\widehat{\phi})$ so
the result follows.
\end{proof}


\section{Modular polynomials}
This section will serve as an introduction to modular polynomials.
I will be following \cite{Lang2} and \cite{AAEC}, and I refer to those for a more thorough examination
with proofs.

These polynomials get their name from the theory of modular functions. That is a topic which is
well beyond the scope of this article, but they are functions invariant under some fractional linear
transform. 

Given a matrix
$$ \gamma = \begin{pmatrix}
 a & b \\ c & d
\end{pmatrix} \in SL_2(\mathbb{Z})$$
we can define an action of $\gamma$ on some $\tau \in \mathbb{C}$ as
$$\gamma(\tau) = \frac{a\tau + b}{c\tau + d} $$

Recall that an elliptic curve over $\mathbb{C}$ is isomorphic to a lattice
$$\Lambda = \mathbb{Z}\omega_1 + \mathbb{Z}\omega_2$$ in $\mathbb{C}$.
Letting $\tau = \frac{\omega_1}{\omega_2}$ we consider the $j$-invariant as a function
on the upper half-plane. $j(\tau)$ is the $j$-invariant of the curve given by such a lattice.

It can be shown that $j$ is a modular function of weight $0$ satisfying
$$ j(\tau) = j(\tau + 1) \quad \text{and} \quad j(\tau) = j(-\frac{1}{\tau}) $$
These are exactly the transformations given by the matrices
$$ \begin{pmatrix} 1 & 1 \\ 0 & 1 \end{pmatrix} \quad \text{and} \quad
   \begin{pmatrix} 0 & -1 \\ 1 & 0 \end{pmatrix} $$
It can be shown that these matrices generate the modular group $SL_2(\mathbb{Z})$,
so we have that
$$j(\tau) = j(\alpha \tau) \quad \alpha \in SL_2(\mathbb{Z}) $$

We are not so much interested in how $j$ stays invariant under the modular group,
but rather how it is acted upon by matrices of the bigger group $GL_2(\mathbb{Z})$.
We thus define
$$ j \alpha(\tau) = j\left(\frac{a\tau + b}{c\tau + d}\right) $$
to be the $j$-invariant of the curve given by the lattice $\mathbb{Z}+\mathbb{Z}\tau'$
with $$\tau' = \frac{a\tau + b}{c\tau + d}$$

Letting $n$ be a positive integer we define a subgroup
$$ S_n^* = \{ \alpha = \begin{pmatrix} a & b \\ 0 & d \end{pmatrix} | \det(\alpha)=n,\, \gcd(a,b,d)=1,\, 0 \leq b < d \} \subset GL_2(\mathbb{Z}) $$
It can be shown that if $n = \ell$ a prime number we have that $\#S_n^* = \ell + 1$.

\begin{mydef}
 The \emph{modular polynomial of degree $\ell$} is given by
$$\Phi_\ell(x,j) = \Phi_\ell(j,x) = \prod_{\alpha \in S_\ell^*}(x - j \alpha) $$
and is of degree $\ell-1$.
\end{mydef}
Notice that the roots of the above polynomials is by definition some $j'$ which is one of transformations
of $j$ under the subgroup $S_\ell^*$. The next theorem gives us a connection between the roots of the
modular polynomial and isogenies between elliptic curves, a proof can be found in \cite{Lang2}.

\begin{thm} \label{modpol}
 Let $E_1$ and $E_2$ be two elliptic curves with $j$-invariants $j(E_1)$ and $j(E_2)$ respectively.
Then there exits an isogeny $f: E_1 \rightarrow E_2$ with $\ker(f)$ cyclic of size $\ell$ if and only if
$$\Phi_\ell(j(E_1), j(E_2)) = 0 $$
\end{thm}

This theorem holds in any field of characteristic $0$ and for a finite field of characteristic $p$ where
$\ell \neq p$.

\section{Schoof's algorithm and improvements}
\subsection{Division polynomials}
The idea of Schoof's algorithm is to calculate the Frobenius trace modulo small primes,
then assemble this information using the Chinese remainder theorem. Chosing the set
of small primes such that their product $N > 4 \sqrt q$ (with $q$ the size of our
field) gives us the trace $t$ modulo $N$, which by the Hasse bound is exactly
the Frobenius trace.

Recall for this section that an elliptic curve corresponds to a lattice $\Lambda$
so we have an isomorphism
$$ \bar{k}/\Lambda \simeq E(\bar{k}) $$
$$ z \mapsto (\wp(z), \wp '(z)) $$
where $\wp(z)$ is the elliptic Weierstrass function
$$\wp(z) = \frac{1}{z^2} + \sum_{k=1}^\infty c_k z^{2k} $$

\begin{mydef}
 The \emph{division polynomials} are polynomials $\Psi_n(x,y) \in \mathbb{Z}[x,y,A,B]$
defined by the recurrence relations
\begin{align*}
  \Psi_0 &= 0 \\
  \Psi_1 &= 1 \\
  \Psi_2 &= 2y \\
  \Psi_3 &= 3x^4 + 6Ax^2 + 12Bx - A^2 \\
  \Psi_{2n+1} &= \Psi_{n+2} \Psi_n^3 - \Psi_{n+1}^3 \Psi_{n-1} \\
  \Psi_{2n}   &= (2y)^{-1} \Psi_n(\Psi_{n+2} \Psi_{n-1}^2 - \Psi_{n-2} \Psi_{n+1}^2)
\end{align*}
where $\Psi_n(x,y) = 0$ is and only if $(x,y) \in E[n]$.
\end{mydef}
The construction of these polynomials can be done in at least two ways and I will discuss
both of them briefly.

One way of doing this is to construct
a function having poles at the $n$-torsion points of our elliptic curve as follows
$$ f_n(z) = n^2 \prod(\wp(z) - \wp(u)) $$
where the product is taken over all $n$-torsion points of $\bar{k}/\Lambda$, denoted
$\bar{k}/\Lambda[n]$. This function has roots at exactly the $n$-torsion points by definition,
which is at least what we want. A more throrough examination of this method can be found
in \cite{Serge}.
Another way which is more elementary but highly computational is to work explicitly
with the addition formulas for elliptic curves.

Replacing the terms $y^2$ in $\Psi_n$ by $x^3 + Ax + B$ we obtain polynomials $\Psi_n '$ in
$\mathbb{F}_q[x]$ if is $n$ is odd or $y \mathbb{F}_q[x]$ if $n$ is even. To avoid
this distinction we define
$$
f_n(x) = \begin{cases}
          \Psi_n '(x,y) & \text{if n is odd} \\
	  \Psi_n '(x,y)/y & \text{if n is even}
         \end{cases}
$$


\begin{prop}
 Let $n \geq 2$ and $\Psi_n$ the division polynomial as defined above, then
$$ nP = (x - \frac{\Psi_{n-1} \Psi_{n+1}}{\Psi_n^2}, \frac{\Psi_{n+2} \Psi_{n-1}^2 - \Psi_{n-2} \Psi_{n+1}^2}{4y \Psi_n^3} )$$
\end{prop}

\subsection{Schoof's algorithm}
For an elliptic curve over $\mathbb{F}_q$ given by
$$ E: y^2 = x^3 + Ax + B $$
we want to compute the size of $\#E(\mathbb{F}_q)$, we know from before that
$$ \#E(\mathbb{F}_q) = q + 1 - t $$
where $t$ is the trace of the Frobenius as seen in chapter \ref{frob}. We know
that $t$ satisfies the Hasse bound namely
$$ |\#E(\mathbb{F}_q)-q-1|=|t| < 2\sqrt{q} $$
Let $S = \{3, 5, 7, 11, \ldots \, L \}$ be the set of odd primes $\leq L$ such
that the product is bigger than the Hasse interval
$$ N = \prod_{\ell \in S} \ell  > 4\sqrt{q} $$
If we can then calculate $t\, (\mod \ell)$ for all $\ell \in S$ we can uniquely
determine $t\,(\mod N)$ by invoking the Chinese remainder theorem,
which then by the Hasse bound is our Frobenius trace $t$.

We will now look at
how to calculate $t\, (\mod \ell)$. Let $\phi$ be the Frobenius endomorphism
resticted to $E[\ell]$ and let $q_\ell$, $\tau$ be $q$ and $t$ reduced modulo $\ell$
respectively. The computation of $\tau$ can then be done by checking if
$$ \phi^2(P) + q_\ell P = \tau \phi(P) $$
holds for $P \in E[\ell]$. To perform the addition on the left hand side of the
equality we need to distinguish the cases where the points are on a vertical line or not.
In other words we have to verify if for $P = (x,y) \in E[\ell]$ the following holds
$$ \phi^2 (P) = \pm q_\ell P $$
Noting that $-P = (x, -y)$ we write out the equality for the $x$-coordinates in terms of
division polynomials
$$ x^{q^2} = x - \frac{\Psi_{q_\ell-1} \Psi_{q_\ell+1}}{\Psi_{q_\ell}^2}(x,y) $$
Writing this out in terms of $f_n(x)$ and noting that for $n$ even we have
$\Psi_n(x,y) = y f_n(x)$, a calculation for $q_\ell$ even yields
\begin{eqnarray*}
 x^{q^2} &=& \frac{f_{q_\ell-1}(x) f_{q_\ell+1}(x)}{(f_{q_\ell} y)^2} \nonumber \\
	 &=& \frac{f_{q_\ell-1}(x) f_{q_\ell+1}(x)}{f_{q_\ell}^2 (x^3+Ax+B)} \nonumber \\
\end{eqnarray*}
The calculation for $q_\ell$ odd is similar and we get the equality

$$
x^{q^2} = \begin{cases}
           x - \frac{f_{q_\ell-1}(x) f_{q_\ell+1}(x)}{f_{q_\ell}^2 (x^3+Ax+B)} & \text{if } q_\ell \text{ is even} \\
	   x - \frac{f_{q_\ell-1}(x) f_{q_\ell+1}(x) (x^3+Ax+B)}{f_{q_\ell}^2(x)} & \text{if } q_\ell \text{ is odd} 
          \end{cases}
$$
We thus get two equations and we want to verify they have any solutions $P \in E[\ell]$. For
doing this we compute the following greatest common divisors
$$ \gcd((x^{q^2} - x)f_{q_\ell}^2 (x^3+Ax+B)+f_{q_\ell-1}(x) f_{q_\ell+1}(x), f_\ell(x)) \quad (q_\ell \text{ even)}$$
$$ \gcd((x^{q^2} - x)f_{q_\ell}^2(x)+f_{q_\ell-1}(x) f_{q_\ell+1}(x) (x^3+Ax+B), f_\ell(x)) \quad (q_\ell \text{ odd)}$$
We are now going to treat the rest in two cases, depending on the value from the above gcds.

\textbf{Case 1:} $\gcd \neq 1$ meaning there exist a non-zero $\ell$-torsion point $P$ such that $\phi^2(P) = \pm q_\ell P$.
If $\phi^2 (P) = -q_\ell P$ we have that $\tau \phi(P) = 0$ but since $\phi(P) \neq 0$ we know that $\tau = 0$.
If $\phi^2(P) = q_\ell P$ we have that 
$$ 2 q_\ell P = \tau \phi(P) \Leftrightarrow \phi(P) = \frac{2 q_\ell}{\tau} $$
Substituting the last equality into $\phi^2(P) = q_\ell P$ we obtain
$$ \frac{4 q_\ell^2}{\tau^2} = q_\ell P \Leftrightarrow 4 q_\ell P = \tau^2 P $$
We thus obtain the congruence $\tau^2 \equiv 4q_\ell \quad (mod \ell)$

\textbf{Case 2:} $\gcd = 1$ so $\phi^2(P) \neq \pm q_\ell P$ meaning the two points are 
not equal nor are they on the same vertical line for any $\ell$-torsion point $P$. 
This enables us to do the addition $\phi^2(P) + q_\ell P$ using the appropriate addition formulas.
Recall that if $P = (x_1, y_1)$ and $Q = (x_2, y_2)$ are two points on $E$ with
$P \neq Q$ we have that their sum is given by $P+Q = (x_3, y_3)$ where
$$ \lambda = \frac{y_2 - y_1}{x_2 - x_1} $$
$$ x_3 = -x_1 - x_2 + \lambda^2 $$
$$ y_3 = -y_1 -\lambda(x_3 - x_1) $$
We can now write out the addition explicitly in terms of polynomials as follows
$$ \lambda = \frac{\Psi_{q_\ell+2} \Psi_{q_\ell-1}^2 - \Psi_{q_\ell-2}\Psi_{q_\ell+1}^2 - 4y^{q^2+1}\Psi_{q_\ell}^3}
		  {4\Psi_{q_\ell} y ((x-x^{q^2})\Psi_{q_\ell}^2 - \Psi_{q_\ell-1}\Psi_{q_\ell+1}} $$
with the left hand side given by

$$\phi^2 (P) + q_\ell P = \left(-x^{q^2}-x+\frac{\Psi_{q_\ell-1}\Psi_{q_\ell+1}}{\Psi_{q_\ell}^2}+\lambda^2,
		     -y^{q^2}-\lambda\left(-2x^{q^2}-x+\frac{\Psi_{q_\ell-1}\Psi_{q_\ell+1}}{\Psi_{q_\ell}^2}\right)\right)$$
The right hand side is as before given by
$$ \tau\phi(P)=\left(x^q-\left(\frac{\Psi_{\tau+1}\Psi_{\tau-1}}{\Psi_\tau^2}\right)^q,\left(\frac{\Psi_{\tau+2}\Psi_{\tau-1}^2 - \Psi_{\tau-2}\Psi_{\tau+1}^2}{4y\Psi_\tau^3}\right)^q\right) $$
So figuring out if $$\phi^2(P) + q_\ell P = \tau \phi(P) $$ amounts to checking if the above equalities
hold for $P\in E[\ell]$ and $0 \leq \tau < \ell$, working modulo the division polynomials $\Psi_\ell(x)$. 
\subsection{Schoof-Elkies algorithm}
When doing the calculations in Schoof's algorithm we were working modulo the division
polynomials $\Psi(x,y)$ of degree $\ell^2-1$. Instead we can exploit some special primes
called \emph{Elkies primes} that enables us to work in a cyclic subgroup $C$ of $E[\ell]$.
Here $C$ will correspond to a $1$-dimensional eigenspace.

The frobenius endomorphism restricted to $E[\ell]$ satisfies the characteristic equation
$$ \phi^2 - \tau \phi + q_\ell = 0 $$
where $\tau$ and $q_\ell$ is as before. The roots of this equations are the eigenvalues
of $\phi | E[\ell]$ and they are given by
$$ \lambda_{1,2} = \frac{\tau \pm \sqrt{\tau^2 - 4 q_\ell}}{2} $$
If the discriminant $\tau^2 - 4 q_\ell$ is a square modulo $\ell$ we have that
$\lambda_{1,2} \in \mathbb{F}_q$.
\begin{mydef}
A prime $\ell$ such that $\tau^2 - 4 q_\ell$ is a square modulo $\ell$ is called
an \emph{Elkies prime}.
\end{mydef}
For primes of this type we obtain a factorization
$$ (\phi - \lambda_1)(\phi - \lambda_2) = 0 $$
so for an eigenvalue $\lambda$ we have that $\phi(P) = \lambda P$ for a point $P$. Thus
$P$ is the generator for a cyclic eigenspace $C \subset E[\ell]$ of order $\ell$
corresponding to $\lambda$.
Notice that we have an exact sequence of groups
$$ 0 \rightarrow C \rightarrow E \rightarrow E/C \rightarrow 0 $$
where the map $E \rightarrow E/C$ has cyclic kernel $C$ of order $\ell$.
Determining which primes are Elkies primes can be done by working with the
modular polynomials. From [ref til modulærpoly-teorem] we have that $\Phi_\ell(j(E),j(E/C)) = 0$,
so letting the isogeny $f: E \rightarrow E'$ have cyclic kernel $C$ we get an exact
sequence
$$ 0 \rightarrow C \rightarrow E \rightarrow E' \rightarrow 0 $$
which by a diagram chase yields $ E' \simeq E/C $. This argument gives us the following result
\begin{prop}
 $ \Phi_\ell(j(E), x) = 0$ for $x \in \mathbb{F}_q$ if and only if $\ell$ is an Elkies
prime.
\end{prop}
Figuring out if $\ell$ is an Elkies prime can thus be done fast by calculating
$$\gcd(\Phi_\ell(j(E), x), x^q - x)$$
Now since we are working only with primes of this type we restrict ourself to working
in the subspace $C$ of order $\ell$. There is thus a factor $G_\ell(x)$ of the division polynomial
which has the $x$-coordinates of points in $C$ as roots. Since
similar points in $C$ of different sign are on the same vertical line we only include
unique points up to sign. In this way we get that the degree of $G_\ell(x)$ is $\frac{\ell-1}{2}$.

From the theory of eigenvalues we know that if $\lambda_1, \lambda_2$ are
eigenvalues of $\phi$ then $$tr(\phi) = \lambda_1 + \lambda_2$$
We also know using proposition \ref{detdeg} that $$\lambda_1 \lambda_2 = det(\phi) = q $$
Using this we can recover the trace of $\phi$ by calculating one of the eigenvalues
$$ \tau \equiv \lambda + \frac{q}{\lambda} \quad (mod\, \ell) $$
To compute the eigenvalue $\lambda$ we can thus check which of the relations
$$ \phi(P) = (x^q, y^q) = \lambda P$$
holds on the eigenspace $C$, this mean we can work modulo $G_\ell(x)$. This enables us
to work in the much smaller ring $$\mathbb{F}_q[x,y]/(G(x), y^2 - x^3 - Ax - B) $$
and thus greatly improves Schoof's original approach.

The obstacle that remains is how we can possible calculate the factor 
$$ G_\ell(x) = \prod_{(x',y')\in C} (x-x')$$
of the division polynomialwhere the product is taken over all 
unique points $P = (x', y')$ up to sign. When calculating the gcd
$\gcd(\Phi_\ell(j(E), x), x^q - x)$ we obtain a polynomial whos roots
(at most two) are the $j$-invariants of the $\ell$-isogenous curves
$\tilde{E} = E/C$ where $C$ is the eigenspace corresponding to $\lambda$.
The next theorem enables us to calculate an explicit formula for the Weierstrass
equation of $\tilde{E}$.

\begin{thm}
 Let $E$ be given by the equation
$$ E: y^2 = x^3 + Ax + B $$
with $j = j(E)$. Then the eqation for the $\ell$-isogenous curve $\tilde{E}$ with
$\tilde{j} = j(\tilde{E})$ is given by
$$ \tilde{E}: y^2 = x^3 + \bar{A}x + \bar{B} $$

$$\bar{A} = -\frac{\tilde{j}'^2}{48 \tilde{j}(\tilde{j} - 1728)} \quad
  \bar{B} = -\frac{\tilde{j}'^3}{864 \tilde{j}^2(\tilde{j} - 1728)} $$

And letting $$\Phi_{\ell, x} = \frac{\partial \Phi_\ell}{\partial x} \quad
              \Phi_{\ell, y} = \frac{\partial \Phi_\ell}{\partial y}$$
be the partial derivatives with respect to $x$ and $y$ respectively we have that
$$ \bar{j}' = -\frac{18 B \Phi_{\ell, x}(j, \bar{j})}{\ell A \Phi_{\ell, y}(j, \bar{j})} j $$
\end{thm}
The next theorem will enable us to compute the sum of the $x$-coordinates of the points in
our subspace $C$. This value will be used to calculate every coefficient of $G_\ell(x)$, notice
that if we formally multiply out the product of $G_\ell(x)$ we get
$$ G_\ell(x) = x^\frac{\ell-1}{2} - \frac{p_1}{2} x^\frac{\ell-3}{2} + \ldots $$
Here $p_1$ is the sum of the $x$-coordinates of $C$, the division by two is because they
appear twice as a result of the symmetry around the $x$-axis.
\begin{thm}
 Given our two curves $E: y^2 = x^3 + Ax + B$ and $\tilde{E}: y^2 = x^3 + \bar{A}x + \bar{B}$ we let
$E_4 = -48A$, $E_6 = 864B$ and similarly for our $\ell$-isogenous curve $\bar{E_4} = -48\bar{A}$,
$\bar{E_6} = 864\bar{B}$. Then we obtain an explicit formula for
$$ p_1 = \sum_{(x,y)\in C} x $$
$$ p_1 = \frac{\ell}{2}J + \frac{\ell}{4} \left( \frac{E_4^2}{E_6} - \ell \frac{\bar{E_4^2}}{\bar{E_6}} \right) $$
where by using the usual partial derivative notation $\Psi_{\ell, xx} = \frac{\partial^2 \Psi_\ell}{\partial x^2}$
etc. we write $J$ as 
$$ J = -\frac{j'^2 \Psi_{\ell, xx}(j, \tilde{j}) + 2\ell j' \tilde{j'} \Psi_{\ell, xy}(j, \tilde{j})
+\ell^2 \tilde{j'}^2 \Psi_{\ell, yy}(j, j')}{j' \Psi_{\ell, x}(j, j')} $$
Here $j'= -j \frac{E_6}{E_4}$ and similarly $\tilde{j'} = -\tilde{j} \frac{\tilde{E_6}}{\tilde{E_4}}$.
\end{thm}

Given the value of $p_1$ we can calculate the rest of the coefficients using a theorem from [schoof-ref].
\begin{thm}
 Let $\ell \neq char(k)$ be a prime, and $\phi: E \rightarrow \widetilde{E}$ be an isogeny with $\ker(\phi)$
cyclic of size $\ell$, then the polynomial $G_\ell(x)$ which vanishes on the $x$-coordinates of
elements of $\ker(\phi)$ satisfies
$$ z^{\ell-1}G_\ell(\wp(z)) = exp(-\frac{1}{2}p_1 z^2 - \sum_{k=1}^\infty \frac{\widetilde{c}_k-\ell c_k}{(2k+1)(2k+2)}z^{2k+2})$$
\end{thm}
\begin{proof}
 The proof of Schoof uses analytic theory heavily, he introduces the Weierstrass $\zeta$-function
which is defined by
$$\zeta(z) = \frac{1}{z} - \sum_{k=1}^\infty \frac{c_k}{2k+1} z^{2k+1} $$
Differentiating we see that $\zeta'(z) = - \wp(z)$. Let $\Lambda = \omega_1 \mathbb{Z}+\omega_2\mathbb{Z}$
be the lattice corresponding to $E$ and then we have that
$\widetilde{\Lambda} = \frac{\omega_1}{\ell}\mathbb{Z}+\omega_2\mathbb{Z}$ is the lattice corresponding
to $\widetilde{E}$. Setting $\zeta$ and $\widetilde{\zeta}$ to be the Weierstrass $\zeta$-functions for
$E$ and $\widetilde{E}$ respectively, Schoof eventually arrives at the equality
$$-\ell \zeta(z)+\widetilde{\zeta}(z)-p_1 z = \sum_{i=1}^{(\ell-1)/2} \frac{\wp'(z)}{\wp(z)-\wp(\frac{i}{\ell}\omega_1)} $$
Notice that $\frac{d}{dz} \wp(z) - \wp(\frac{i}{\ell}\omega_1) = \wp'(z)$ so we can invert the process
of logaritmic differention
$$ \frac{df}{dz} = \frac{f'}{f}$$
on both sides of the equality, setting $$f(z) = \prod_{i=1}^{(\ell-1)/2}(\wp(z)-\wp(\frac{i}{\ell}\omega_1)$$

A thurough proof can be found in [schoof-ref].
\end{proof}

The coefficients of $G_\ell(x)$ can thus be obtained by expanding both sides of
the equality from the previous theorem and comparing the coefficients of like powers of $z$.
Setting $w=z^2$ and letting $A(w)$ be the function on the right-hand side of the equality
expanded as a power series in $w$. Also let $C(w) = \wp(z) - \frac{1}{w} = \sum_{k=1}^\infty c_k w^k$,
the Weierstrass $\wp$-function with the first term removed. For notational convenience we
write $[B(w)]_j$ for the coefficient of $w^j$ in the power series $B(w)$. Letting $g_i$ be the
coefficient of $x^i$ in $G_\ell(x) = x^d + \sum_{i=0}^{d-1} g_i x^i$ with $d = \frac{\ell-1}{2}$ 
we get the recursion
$$g_{d-i} = [A(w)]_i - \sum_{k=1}^i \left( \sum_{j=0}^k \binom{d-i+k}{k-j} [C(w)^{k-j}]_j \right) g_{d-i+k} $$

\section{Satoh's algorithm} \label{satoh}
Forklaring og overblikk over de forskjellige subsection-ene.

This algorithm is divided into two parts, first we do what is called a \emph{lifting}, then
we recover the trace of the Frobenius from the lifted data.

\subsection{Lifting the j-invariants}
We begin by establishing some notation, so let $\mathbb{F}_q$ be our finite field with $q=p^n$ as before,
$\mathbb{Z}_p$ the $p$-adic integers and $\mathbb{Q}_q$ the $q$-adic rationals as defined in section \ref{p-adic}.
For this section we let $\sigma$ be the $p$-th frobenius, and $\phi_q$ be the $q$-th frobenius.
As for previous sections we denote the curves over our finite fields as $E/\mathbb{F}_q$,
for the lifted curves we write $\mathscr{E}/\mathbb{Q}_q$.

\begin{thm}
 \textbf{(Lubin-Serre-Tate)} Let $E/\mathbb{F}_q$ be an elliptic curve with $j$-invariant $j(E)$ and
$\sigma$ the $p$-th Frobenius on $\mathbb{Q}_q$ then the system of equations
$$ \Phi_p(x, \sigma(x)) = 0 \quad x \equiv j(E) \, (mod\, p)$$
where $\Phi_p$ is the $p$-th modular polynomial has a unique solution $J \in \mathbb{Z}_q$ 
which is the $j$-invariant of the canonical lift $\mathscr{E}$ of $E$.
\end{thm}
The latter theorem gives an efficient way of calculating the $j$-invariants, in addition it has
been shown \cite{Deuring} that the canonical lift always exists and is unique (up to isomorphism).

Knowing $j(\mathscr{E})$ we can explicitly write out the Weierstrass equation for $\mathscr{E}$, but
instead of lifting $E$ to $\mathscr{E}$ directly we can consider all its conjugates
$$E, E^\sigma, E^{\sigma^2}, \ldots, E^{\sigma^{n-2}}, E^{\sigma^{n-1}} $$
Letting $E^{\sigma^i} = E^i $ we get a sequence of maps
$$ E \overset{\sigma}{\rightarrow} E^1 \overset{\sigma}{\rightarrow} E^2 \overset{\sigma}{\rightarrow}
\ldots \overset{\sigma}{\rightarrow} E^{n-1} $$
Where the composition is the $q$-th power Frobenius $\phi_q = \sigma \sigma \ldots \sigma: E \rightarrow E$.
Recall that the $deg(\sigma) = p$ so from the theory of modular polynomials we have that
$$ \Phi_p(j(E^i), j(E^{i+1})) = 0 $$

\begin{mydef}
 The \emph{canonical lift $\mathscr{E}$} of an elliptic curve $E$ over $\mathbb{F}_q$ is
an elliptic curve over $\mathbb{Q}_q$ such that $End(\mathscr{E}) \simeq End(E)$.
\end{mydef}

Since the endomorphism rings are isomorphic we can lift every Frobenius on $E$ to a
Frobenius on $\mathscr{E}$. We thus obtain a commutative diagram

$$
\xymatrix {
  \mathscr{E} \ar[d]^\pi \ar[r]^\sigma & \mathscr{E}^1 \ar[d]^\pi \ar[r]^\sigma & \ldots \ar[r]^\sigma & \mathscr{E}^{n-1} \ar[d]^\pi \ar[r]^\sigma & \mathscr{E} \ar[d]^\pi \\
  E \ar[r]^\sigma & E^1 \ar[r]^\sigma &\ldots \ar[r]^\sigma & E^{n-1} \ar[r]^\sigma & E \\
}
$$

Since the lifted Frobenius also has degree $p$ we have that
$$\Phi_p(j(\mathscr{E}^i), j(\mathscr{E}^{i+1})) = 0 \quad j(\mathscr{E}^i) \equiv j(\mathscr{E}^{i+1}) \, (mod\, p) $$
We thus define a function $\Theta: \mathbb{Z}_q^d \rightarrow \mathbb{Z}_q^d$ by
$$\Theta(x_0, x_1, \ldots, x_{n-1}) = (\Phi_p(x_0, x_1), \Phi_p(x_1, x_2), \ldots, \Phi_p(x_{n-1}, x_0))$$
Note that the roots of $\Theta$ are the $j$-invariants of our lifted curves
$$\Theta(j(\mathscr{E}), j(\mathscr{E}^2), \ldots, j(\mathscr{E}^{n-1})) = (0, 0, \ldots, 0) $$
so by solving $\Theta(\bar{x}) = 0$ using a multivariate Newton-Raphson iteration, we can
recover the $j$-invariants to desired precision. Setting up the Jacobian matrix $J_\Theta$
of $\Theta$, the iteration is given by
$$ \bar{x}_{n+1} = \bar{x}_n - J_\Theta^{-1} \Theta(\bar{x}_n) $$
where the matrix $J_\Theta(x_0, x_1, \ldots, x_{n-1})$ is given as 

$$
\begin{pmatrix}
  {\partial \over \partial x_0} \Psi_p(x_0, x_1) & {\partial \over \partial x_1} \Psi_p(x_0, x_1) & 0 & \ldots & 0 & 0 \\
  0 & {\partial \over \partial x_1} \Psi_p(x_1, x_2) & {\partial \over \partial x_2} \Psi_p(x_1, x_2) & 0 & \ldots & 0 \\
  0 \\
  \vdots & & \ddots & & & \vdots \\
  0 \\
  {\partial \over \partial x_0} \Psi_p(x_{n-1}, x_0) & 0 & \ldots & 0 & 0 & {\partial \over \partial x_{n-1}} \Psi_p(x_{n-1}, x_0) \\
\end{pmatrix}
$$

\subsection{Recovering the trace}
Let $\phi$ be the $q$-th Frobenius and $\phi^*$ be the induced Frobenius on differentials, 
we have that $c = Tr(\phi) = \phi + \hat{\phi}$ so investigating the
action of the Frobenius on the invariant differential $\omega$ we see that
\begin{eqnarray}
 [Tr(\phi)]^*(\omega)&=& [Tr(\phi)](\omega) \nonumber \\
		     &=& (\phi + \hat{\phi})^*(\omega) \nonumber \\
		     &=& \phi^*(\omega) + \hat{\phi}^*(\omega) \nonumber \\
		     &=& \hat{\phi}^*(\omega) \nonumber
\end{eqnarray}
Where the last equality is using the fact that $\phi^* = 0$ since $\phi$ is inseperable, we thus
get that $\hat{\phi}^*(\omega) = c\omega$.
Recall from \ref{invariant} that $\frac{dx}{y}$ is also holomorphic and invariant under translation,
so for the rest of this section we define our invariant differential as such
$$ \omega = \frac{dx}{y} $$
Instead of working with $\phi$ we work with its dual $\hat{\phi}$ and the dual
of the $p$-th Frobenius $\hat{\sigma}$. Our diagrams will be turned around so we get commutative
squares
$$
\xymatrix {
  \mathscr{E}^{i+1} \ar[r]^{\hat{\sigma}_{i+1}} \ar[d]^\pi & \mathscr{E}^i \ar[d]^\pi \\
  E^{i+1} \ar[r]^{\hat{\sigma}_{i+1}} & E^i 
}
$$

Letting $\hat{\mathscr{F}_q}$ be the lifted of the dual $q$-th Frobenius we have that
$\hat{\mathscr{F}_q} = \hat{\sigma} \hat{\sigma} \ldots \hat{\sigma}$.
So if $\omega_i = \omega^{\sigma^i}$ we have that $\hat{\sigma}_i^*(\omega_i) = c_i \omega_{i+1}$.
A calculation then yields, using that $\sigma_i^* = c_i$
\begin{eqnarray}
  \hat{\mathscr{F}}_q(\omega) &=& (\hat{\sigma}_1 \circ \hat{\sigma}_2 \circ \ldots \circ \hat{\sigma}_{n-1})(\omega) \nonumber \\
			      &=& ([c_1] \circ \ldots \circ [c_{n-1}](\omega) \nonumber \\
			      &=& [c_1\ldots c_{n-1}](\omega) \nonumber
\end{eqnarray}
Since $\hat{\mathscr{F}}_q(\omega) = c \omega$ we have that
$$ c = \prod_{i=1}^{n-1} c_i \quad (mod\, q) $$
It then remains for us to calculate each $c_i$ for every lifted $p$-th Frobenius
endomorphism $\hat{\sigma}_i$.

From \cite{AEC} we have that there exists a commutative triangle
$$
\xymatrix{
  \mathscr{E}_{i+1} \ar[rr]^{\widehat{\sigma}_{i+1}} \ar[dr]^{v_{i}} && \mathscr{E}_i \\
  & \mathscr{E}_{i+1}/\ker(\widehat{\sigma}_{i+1}) \ar[ur]^{\lambda_i} & 
}
$$

From formulas due to V\'{e}lu (see \cite{Velu} or \cite{Sato}) we can calculate the map
$v_i$ and the Weierstrass equation for the curve $\mathscr{E}_{i+1}/\ker(\widehat{\sigma}_{i+1})$.
This means that in order to investigate the action of
$\widehat{\sigma}_{i+1}$ on the invariant differential for all $i$ amount to investigating
how the composition $\lambda_i v_i$ acts. In addition, if we let $v_i^*$ be the map induced 
on differentials then by the formulas of Velu it has trivial action on the invariant differential $\omega$.
It is then enough to calculate how the isomorphism $\lambda_i$ acts on the invariant differential. 
Given the Weierstrass equations for our curves
$$ \mathscr{E}_{i+1}/\ker(\widehat{\sigma}_{i+1}): y^2 = x^3 + \alpha_{i+1}x + \beta_{i+1}$$
$$ \mathscr{E}_i: y^2 = x^3 + a_i x + b_i $$
$$ \lambda_i: \mathscr{E}_{i+1}/\ker(\widehat{\sigma}_{i+1}) \rightarrow \mathscr{E}_i $$
Here the coefficients $\alpha_{i+1}$ and $\beta_{i+1}$ are given by
$$\alpha_{i+1} = (6-5p)a_{i+1}-30(h_{i,1}^2-2h_{i,2})$$
$$\beta_{i+1}  = (15-14p)b_{i+1}-70(3h_{i,1}h_{i,2}-h_{i,1}^3-3h_{i,3})+42a_{i+1}h_{i,1}$$
where $h_{i,k}$ is the coefficient of $x^{d-k}$ in the polynomial $H_i(x)$ from \ref{satohdiv}.

The function which preserves the coefficients of the curves is given by
$$(x,y) \mapsto (u_i^2 x, u_i^3 y) $$
Calculating how this acts on the curve we get the curve
$$ y^2 = x^3 + u_i^{-4} a_i x + u_i^{-6} b_i$$
comparing coefficients we get the two equalities
$$ u_i^{-4} a_i = \alpha_{i+1} \text{ and } u_i^{-6} b_i = \beta_{i+1} $$
Solving for $u_i^2$ we get
$$ u_i^2 = \frac{\alpha_{i+1} b_i}{\beta_{i+1} a_i} $$
and we have our isomorphism. Now for calculating how $\lambda_i$ acts on the holomorphic
differential $\omega=\frac{dx}{y}$ we recall from \ref{diff} and calculate
\begin{eqnarray}
 \lambda_i^*(\frac{1}{y} dx) &=& \lambda_i^*(\frac{1}{y}) d(\lambda_i^*(x)) \nonumber \\
			     &=& \frac{1}{u_i^3 y} d(u_i^2 x) \nonumber \\
			     &=& \frac{u_i^2 dx}{u_i^3 y} \nonumber \\
			     &=& u_i^{-1} \omega \nonumber
\end{eqnarray}
From our commutative triangle we thus have that
$$ \widehat{\sigma_i}^*(\omega_i) = c_i = (\lambda_i v_i)^*(\omega_i) = \lambda_i^*(\omega_i) = u_i^{-1}\omega_{i+1}$$
so we have found $c_i$ for all $i$, its square is given by
$$ c_i^2 = \frac{\beta_{i+1} a_i}{\alpha_{i+1} b_i} $$
By our product formula for $c$ we have the the square of $c$ given as
$$ c^2 = \prod_{i=1}^{n-1} c_i = \prod_{i=1}^{n-1} \frac{\beta_{i+1} a_i}{\alpha_{i+1} b_i} $$
Taking the square root we obtain the trace $c$ up to sign.

\subsection{Factor of the division polynomial} \label{satohdiv}
Recall that in order to calculate the curve equation for $\mathscr{E}_{i+1}/\ker(\widehat{\sigma}_{n+1})$
we needed coefficients of the factor
$$H_i(x) = \prod_{(x',y')\in \ker(\widehat{\sigma}_{i+1})} (x-x')$$
of the $p^{th}$ division polynomial $\Psi_{i+1}$. Here the product is taken over all points in
the kernel excluding the identity $O$ and up to sign. Since $\#\ker(\widehat{\sigma}_{i+1}) = p$ we have
that $\deg(H_i(x)) = \frac{p-1}{2}$.

The following result is due to Satoh and serves as a modified Hensel lifting \cite{Robert}, it can be found
in \cite{Satoh} and \cite{Handbook}.
\begin{prop}
 Let $p\geq 3$ be a prime and $\Psi(x) \in \mathbb{Z}_p[x]$ such that $\Psi'(x) \equiv 0\, (mod\, p)$ and
$\Psi'(x) \not\equiv 0\, (mod\, p^2)$. Let $h(x) \in \mathbb{Z}_p[x]$ be a monic polynomial such that
\begin{enumerate}
  \item $h(x) \,(mod\,p)$ is squarefree and relative prime to $\frac{\Psi'(x)}{p}\,(mod\,p)$.
  \item $\Psi(x) \equiv f(x)h(x)\,(mod\,p^{n+1})$
\end{enumerate}
Then the polynomial
$$H(x) = h(x) + \left(\left(\frac{\Psi(x)}{\Psi'(x)} h'(x)\right)\,(mod\, h(x))\right)$$
satisfies $H(x) \equiv h(x) (mod\, p^n)$ and $\Psi(x) \equiv F(x)H(x) (mod \, p^{2n+1})$ for some $F(x)$.
\end{prop}
This gives us the following algorithm, as seen in \cite{Handbook}.
FIXME pseudokode

\bibliographystyle{apalike}
\bibliography{refs}
\nocite{*}

\end{document}